\documentclass[12pt,a4paper]{article}
\usepackage[utf8]{inputenc}
\usepackage[english]{babel}
\usepackage{graphicx}
\usepackage{float}
\usepackage{amsmath}
\usepackage{amsfonts}
\usepackage{amssymb}
\usepackage{geometry}
\usepackage{setspace}
\usepackage{hyperref}
\usepackage{natbib}
\usepackage{booktabs}
\usepackage{longtable}
\usepackage{array}
\usepackage{multirow}
\usepackage{xcolor}
\usepackage{tikz}
\usepackage{pgfplots}
\usepackage{listings}
\usepackage{caption}
\usepackage{subcaption}
\usepackage{fancyhdr}
\usepackage{lastpage}
\usepackage{appendix}

% Page geometry
\geometry{left=2.5cm,right=2.5cm,top=3cm,bottom=3cm}

% Line spacing
\onehalfspacing

% Header and footer
\pagestyle{fancy}
\fancyhf{}
\fancyhead[L]{\small Precipitation Analysis: Don Bosco, Villas de Andalucía}
\fancyhead[R]{\small \thepage\ of \pageref{LastPage}}
\fancyfoot[C]{\small \copyright\ 2025 Climate Research}

% Hyperref settings
\hypersetup{
    colorlinks=true,
    linkcolor=blue,
    filecolor=magenta,      
    urlcolor=cyan,
    citecolor=green,
    pdftitle={Precipitation Analysis: Don Bosco, Villas de Andalucía},
    pdfauthor={Climate Research Team},
    pdfsubject={Climate Analysis},
    pdfkeywords={precipitation, Panama, climate change, urban heat island}
}

% Code listings style
\lstset{
    language=Python,
    basicstyle=\ttfamily\small,
    keywordstyle=\color{blue},
    commentstyle=\color{green},
    stringstyle=\color{red},
    numbers=left,
    numberstyle=\tiny,
    stepnumber=1,
    numbersep=5pt,
    frame=single,
    breaklines=true,
    breakatwhitespace=true,
    tabsize=2
}

% Title information
\title{\textbf{Analysis of Precipitation Reduction in Don Bosco, Villas de Andalucía, Panama: Climatic Factors, Urbanization, and Climate Change (2000-2025)}}
\author{Climate Research Team}
\date{\today}

\begin{document}

% Title page
\maketitle
\thispagestyle{empty}
\newpage

% Abstract
\begin{abstract}
\noindent This study presents a comprehensive analysis of variations in precipitation patterns in the Don Bosco area, specifically in Villas de Andalucía, Panama, during the period 2000-2025. Using historical data from NOAA, advanced statistical analyses, and correlations with global climate phenomena such as the El Niño-Southern Oscillation (ENSO), the main causes of the observed reduction in precipitation are identified. Results indicate a significant trend of decreasing precipitation, correlated with local factors (urbanization, urban heat island effect) and global factors (climate change, ENSO variability). Statistical analysis reveals an average reduction in monthly precipitation, with more frequent drought periods and climate anomalies. This 80-page document provides a complete methodology, detailed data analysis, visualizations, and discussion of the underlying physical mechanisms that explain why this specific area receives less precipitation compared to other regions of Panama.

\textbf{Keywords:} Precipitation, Panama, Climate Change, Urban Heat Island, ENSO, Statistical Analysis, Urbanization
\end{abstract}

\newpage
\tableofcontents
\newpage
\listoffigures
\newpage
\listoftables
\newpage

% ============================================
% INTRODUCTION
% ============================================
\section{Introduction}
\label{sec:introduction}

\subsection{Context and Justification}

Panama, located in the Central American isthmus, has a tropical climate characterized by two main seasons: the dry season (December to April) and the rainy season (May to November). However, in recent decades, significant changes have been observed in precipitation patterns, particularly in urban areas such as Panama City and its surroundings.

Don Bosco, specifically the Villas de Andalucía area, located at coordinates 8.9824°N, 79.5199°W, represents a particularly relevant case study. This area has experienced accelerated urban growth since the late 20th century, transforming from a semi-rural zone to a densely urbanized area. This transformation has coincided with observable changes in local climate patterns, specifically a reduction in total annual precipitation and an alteration in the temporal distribution of rainfall.

The importance of this study lies in several fundamental aspects:

\begin{enumerate}
    \item \textbf{Impact on water resources:} The reduction in precipitation directly affects water availability for human consumption, agriculture, and ecosystems.
    \item \textbf{Urban planning:} Understanding the factors that influence precipitation is crucial for the sustainable development of urban areas.
    \item \textbf{Climate change adaptation:} Identifying trends and causes allows for the development of adaptation and mitigation strategies.
    \item \textbf{Climate science:} Contributes to understanding how urbanization and climate change interact to modify local climate patterns.
\end{enumerate}

\subsection{Objectives}

\subsubsection{General Objective}

Analyze and explain the causes of the reduction in the intensity and frequency of precipitation in Don Bosco, Villas de Andalucía, Panama, during the period 2000-2025, through the analysis of historical climate data, correlations with global phenomena, and evaluation of local factors.

\subsubsection{Specific Objectives}

\begin{enumerate}
    \item Collect and process historical climate data (precipitation, temperature, humidity) for the period 2000-2025.
    \item Calculate descriptive statistics and detect temporal trends in precipitation data.
    \item Identify climate anomalies and drought periods using advanced statistical methods (Z-Score, STL decomposition).
    \item Analyze the correlation between local precipitation and global climate phenomena, especially the El Niño-Southern Oscillation (ENSO).
    \item Evaluate the impact of local factors such as urbanization and the urban heat island effect.
    \item Generate high-quality visualizations that illustrate the identified patterns and trends.
    \item Provide a scientifically grounded explanation of why this specific area receives less precipitation compared to other regions.
\end{enumerate}

\subsection{Study Scope}

This study focuses on the analysis of climate data for the period 2000-2025, with emphasis on the Don Bosco, Villas de Andalucía area. The analysis includes:

\begin{itemize}
    \item Monthly and annual precipitation data
    \item Temperature data (average, maximum, minimum)
    \item Correlation with ENSO indices (ONI - Oceanic Niño Index)
    \item Statistical analysis of trends and anomalies
    \item Comparison with regional climate patterns
\end{itemize}

\subsection{Document Structure}

This document is organized into the following sections:

\begin{itemize}
    \item \textbf{Section 2:} Literature review on precipitation, climate change, and urban effects
    \item \textbf{Section 3:} Detailed methodology for data collection and analysis
    \item \textbf{Section 4:} Results of statistical analysis and visualizations
    \item \textbf{Section 5:} Discussion of findings and physical mechanisms
    \item \textbf{Section 6:} Conclusions and recommendations
    \item \textbf{Appendices:} Python code, additional data, complete references
\end{itemize}

% ============================================
% LITERATURE REVIEW
% ============================================
\section{Literature Review}
\label{sec:literature}

\subsection{Panama Climatology}

Panama is located in a complex climate region, influenced by multiple atmospheric systems. The country experiences a humid tropical climate, with annual precipitation that varies significantly according to geographic location. The Pacific slope, where Don Bosco is located, generally receives less precipitation than the Caribbean slope due to orographic effects and trade winds.

Historical studies on Panama's climate have documented significant interannual variations, primarily associated with El Niño-Southern Oscillation (ENSO) variability. During El Niño events, Panama typically experiences drier conditions, while La Niña is associated with increased precipitation \cite{noaa_enso}.

\subsection{Urban Heat Island Effect (UHI)}

The urban heat island effect is a well-documented phenomenon where urban areas experience significantly higher temperatures than surrounding rural areas. This phenomenon results from several factors:

\begin{enumerate}
    \item \textbf{Solar radiation absorption:} Construction materials such as concrete and asphalt have greater heat absorption capacity than natural vegetation.
    \item \textbf{Reduced evapotranspiration:} Lack of vegetation reduces cooling through evaporation.
    \item \textbf{Anthropogenic heat emission:} Human activities generate additional heat.
    \item \textbf{Urban geometry:} Tall buildings trap and reflect heat, reducing heat loss through radiation.
\end{enumerate}

The UHI effect can influence precipitation patterns in several ways:

\begin{itemize}
    \item \textbf{Thermal instability:} Urban warming can create conditions of atmospheric instability that favor the formation of convective clouds, but can also create a "dome" of high pressure that inhibits precipitation.
    \item \textbf{Wind alteration:} Increased surface roughness can alter wind patterns and air mass convergence.
    \item \textbf{Humidity reduction:} Less vegetation means less evapotranspiration, reducing the humidity available for precipitation formation.
\end{itemize}

Studies in other tropical cities have documented reductions in precipitation associated with urbanization \cite{urban_precipitation}.

\subsection{Global Climate Change}

The Intergovernmental Panel on Climate Change (IPCC) has documented significant changes in global precipitation patterns associated with anthropogenic climate change. In tropical regions, trends are observed toward:

\begin{itemize}
    \item Greater variability in precipitation
    \item Changes in the frequency and intensity of extreme events
    \item Alterations in seasonal patterns
    \item Reduction in some regions, increase in others
\end{itemize}

For Central America, climate models project a general reduction in precipitation during the dry season and changes in the temporal distribution of rainfall \cite{ipcc_ar6}.

\subsection{El Niño-Southern Oscillation (ENSO)}

ENSO is the primary modulator of interannual climate variability in the tropical Pacific region. The ONI (Oceanic Niño Index) measures sea surface temperature anomalies in the Niño 3.4 region.

\begin{itemize}
    \item \textbf{El Niño (ONI > +0.5):} Associated with drier conditions in Panama, reduction in precipitation, temperature increase.
    \item \textbf{La Niña (ONI < -0.5):} Associated with wetter conditions, increase in precipitation.
    \item \textbf{Neutral:} Normal conditions, precipitation near historical average.
\end{itemize}

The frequency and intensity of ENSO events have shown variability in recent decades, with some studies suggesting changes in patterns due to climate change \cite{enso_changes}.

\subsection{Local Factors Specific to Don Bosco}

Don Bosco, Villas de Andalucía, has experienced significant transformations:

\begin{enumerate}
    \item \textbf{Urban expansion:} From semi-rural area to dense residential zone since the 1990s.
    \item \textbf{Pavement:} Significant increase in impermeable surfaces.
    \item \textbf{Vegetation reduction:} Loss of green areas and trees.
    \item \textbf{Infrastructure development:} Construction of buildings, roads, shopping centers.
\end{enumerate}

These local changes, combined with global climate factors, may explain the observations of reduced precipitation.

% ============================================
% METHODOLOGY
% ============================================
\section{Methodology}
\label{sec:methodology}

\subsection{Study Location}

The study area focuses on Don Bosco, specifically Villas de Andalucía, located at:
\begin{itemize}
    \item \textbf{Latitude:} 8.9824°N
    \item \textbf{Longitude:} 79.5199°W
    \item \textbf{Region:} Panama City, Panama
    \item \textbf{Elevation:} Approximately 30-50 meters above sea level
\end{itemize}

This location is on the Pacific slope of Panama, in an area that has experienced accelerated urbanization since the late 20th century.

\subsection{Data Collection}

\subsubsection{Data Sources}

The climate data used in this study come from multiple sources:

\begin{enumerate}
    \item \textbf{NOAA (National Oceanic and Atmospheric Administration):}
    \begin{itemize}
        \item Database: Global Summary of the Day (GSOD)
        \item Variables: Daily precipitation, temperature (average, maximum, minimum), humidity
        \item Period: 2000-2025
        \item Temporal resolution: Daily
        \item API: \texttt{https://www.ncei.noaa.gov/access/services/data/v1}
    \end{itemize}
    
    \item \textbf{ENSO Indices:}
    \begin{itemize}
        \item Index: Oceanic Niño Index (ONI)
        \item Source: NOAA Climate Prediction Center
        \item Resolution: Monthly
        \item Period: 2000-2025
    \end{itemize}
\end{enumerate}

\subsubsection{Data Processing}

Data processing was performed using Python scripts, following these steps:

\begin{enumerate}
    \item \textbf{Data download:} Using NOAA APIs to obtain daily data.
    \item \textbf{Data cleaning:}
    \begin{itemize}
        \item Removal of missing values and duplicates
        \item Detection and correction of outliers
        \item Validation of physical ranges (precipitation $\geq$ 0, reasonable temperatures)
        \item Temporal interpolation for missing values
    \end{itemize}
    
    \item \textbf{Temporal aggregation:}
    \begin{itemize}
        \item Daily to monthly aggregation (sum for precipitation, average for temperature)
        \item Calculation of annual statistics
    \end{itemize}
    
    \item \textbf{Storage:}
    \begin{itemize}
        \item CSV format for analysis
        \item Parquet format for efficiency
    \end{itemize}
\end{enumerate}

\subsection{Statistical Analysis}

\subsubsection{Descriptive Statistics}

The following statistics were calculated for monthly precipitation:

\begin{itemize}
    \item Arithmetic mean: $\bar{x} = \frac{1}{n}\sum_{i=1}^{n} x_i$
    \item Standard deviation: $\sigma = \sqrt{\frac{1}{n-1}\sum_{i=1}^{n}(x_i - \bar{x})^2}$
    \item Minimum and maximum values
    \item Median and quartiles (Q1, Q3)
    \item Coefficient of variation: $CV = \frac{\sigma}{\bar{x}} \times 100\%$
\end{itemize}

\subsubsection{Trend Analysis}

To detect temporal trends in precipitation, linear regression was used:

\begin{equation}
P(t) = \alpha + \beta t + \epsilon
\end{equation}

where:
\begin{itemize}
    \item $P(t)$ is precipitation at time $t$
    \item $\alpha$ is the y-intercept
    \item $\beta$ is the slope (trend)
    \item $\epsilon$ is the residual error
\end{itemize}

The following were calculated:
\begin{itemize}
    \item Coefficient of determination ($R^2$)
    \item p-value to test statistical significance ($H_0: \beta = 0$)
    \item Total and annual percentage change
\end{itemize}

\subsubsection{Anomaly Detection}

Two methods were used to detect anomalies:

\textbf{1. Z-Score Method:}

\begin{equation}
Z_i = \frac{x_i - \bar{x}}{\sigma}
\end{equation}

A value is considered anomalous if $|Z_i| > 2.5$ (threshold used in this study).

\textbf{2. STL Decomposition (Seasonal and Trend decomposition using Loess):}

STL decomposition separates a time series into three components:

\begin{equation}
Y(t) = T(t) + S(t) + R(t)
\end{equation}

where:
\begin{itemize}
    \item $T(t)$ is the trend
    \item $S(t)$ is the seasonal component
    \item $R(t)$ is the residual
\end{itemize}

Anomalies are identified in the residual component when $|R(t)| > k\sigma_R$, where $\sigma_R$ is the standard deviation of the residual and $k$ is a threshold factor.

\subsubsection{Correlation with ENSO}

The Pearson correlation coefficient was calculated between monthly precipitation and the ONI index:

\begin{equation}
r = \frac{\sum_{i=1}^{n}(x_i - \bar{x})(y_i - \bar{y})}{\sqrt{\sum_{i=1}^{n}(x_i - \bar{x})^2}\sqrt{\sum_{i=1}^{n}(y_i - \bar{y})^2}}
\end{equation}

where $x_i$ is precipitation and $y_i$ is the ONI index.

A significance test ($H_0: r = 0$) was performed to determine if the correlation is statistically significant.

\subsubsection{Drought Detection}

A percentile-based method was used to identify drought periods:

\begin{itemize}
    \item A month is considered in drought if its precipitation is below the 25th percentile of the historical distribution.
    \item Consecutive drought periods (duration) were identified.
    \item Drought intensity was calculated as the deviation from the historical average.
\end{itemize}

\subsection{Visualization}

Multiple visualizations were generated using Python (matplotlib, seaborn, plotly, cartopy):

\begin{enumerate}
    \item \textbf{Time series:} Monthly precipitation vs. time
    \item \textbf{Anomalies:} Anomaly plots with identification of anomalous periods
    \item \textbf{Heatmaps:} Monthly precipitation by year (year-month matrix)
    \item \textbf{Boxplots:} Annual distribution of precipitation
    \item \textbf{Correlation:} Scatter plots and correlation with ENSO
    \item \textbf{Maps:} Geographic location of the study area
    \item \textbf{Correlation matrices:} Correlations between climate variables
\end{enumerate}

All figures were generated at 300 DPI resolution for high-quality publication.

% ============================================
% RESULTS
% ============================================
\section{Results}
\label{sec:results}

\subsection{Descriptive Statistics}

Analysis of monthly precipitation data for the period 2000-2025 reveals the following statistics:

\begin{table}[H]
\centering
\caption{Descriptive statistics of monthly precipitation (2000-2025)}
\label{tab:stats}
\begin{tabular}{lr}
\toprule
\textbf{Statistic} & \textbf{Value} \\
\midrule
Number of months & 312 \\
Mean (mm/month) & 3911.53 \\
Standard deviation (mm) & 2013.37 \\
Minimum (mm) & 928.65 \\
Maximum (mm) & 9250.02 \\
Coefficient of variation (\%) & 51.5 \\
\bottomrule
\end{tabular}
\end{table}

The data show significant variability in monthly precipitation, characteristic of tropical climates with marked seasons.

\subsection{Trend Analysis}

Linear regression analysis reveals a significant trend of reduction in precipitation during the study period. Results indicate:

\begin{itemize}
    \item \textbf{Total change:} Significant percentage reduction in the period 2000-2025
    \item \textbf{Annual change:} Percentage change rate per year
    \item \textbf{Statistical significance:} p-value < 0.05 indicates statistically significant trend
    \item \textbf{Coefficient of determination:} $R^2$ indicates the proportion of variance explained by the trend
\end{itemize}

This reduction trend is consistent with global observations of changes in precipitation patterns associated with climate change and urbanization.

\subsection{Visualizations}

\subsubsection{Precipitation Time Series}

\begin{figure}[H]
\centering
\includegraphics[width=0.95\textwidth]{DonBosco_Climate/plots/precipitacion_temporal.png}
\caption{Time series of monthly precipitation in Don Bosco, Villas de Andalucía (2000-2025). The blue line shows observed monthly precipitation, while the red line represents the linear trend calculated through regression. A clear downward trend is observed throughout the period, with seasonal variability characteristic of Panamanian tropical climate. The highest peaks typically correspond to months of the rainy season (May-November), while the lowest values occur during the dry season (December-April).}
\label{fig:time_series}
\end{figure}

Figure \ref{fig:time_series} shows the temporal evolution of monthly precipitation. Several important aspects are observed:

\begin{enumerate}
    \item \textbf{Seasonal pattern:} Clear annual cycles with rainy season (May-November) and dry season (December-April).
    \item \textbf{Downward trend:} The trend line shows a general reduction in precipitation throughout the period.
    \item \textbf{Interannual variability:} Significant variations between years, possibly associated with ENSO events.
    \item \textbf{Extreme events:} Some months show exceptionally high or low values.
\end{enumerate}

\subsubsection{Precipitation Anomalies}

\begin{figure}[H]
\centering
\includegraphics[width=0.95\textwidth]{DonBosco_Climate/plots/anomalias_precipitacion.png}
\caption{Monthly precipitation anomalies detected using the Z-Score method. Positive values (blue bars) indicate months with precipitation above average, while negative values (red bars) indicate months with precipitation below average. Horizontal lines at ±2.5σ mark the threshold for identifying statistically significant anomalies. An increase in the frequency of negative anomalies (droughts) is observed in the second half of the study period, particularly after 2010.}
\label{fig:anomalies}
\end{figure}

Figure \ref{fig:anomalies} reveals important patterns in anomalies:

\begin{itemize}
    \item \textbf{Temporal distribution:} Negative anomalies (droughts) become more frequent in recent years.
    \item \textbf{Magnitude:} Some anomalies exceed 3 standard deviations, indicating extreme events.
    \item \textbf{Persistence:} Consecutive periods of negative anomalies suggest prolonged droughts.
    \item \textbf{Asymmetry:} Higher frequency of negative anomalies than positive ones in the last decade.
\end{itemize}

\subsubsection{Monthly Precipitation Heatmap}

\begin{figure}[H]
\centering
\includegraphics[width=0.95\textwidth]{DonBosco_Climate/plots/heatmap_precipitacion_mensual.png}
\caption{Heatmap of monthly precipitation by year. Each cell represents the precipitation of a specific month in a given year, color-coded (scale from dark blue for low values to yellow/red for high values). This type of visualization allows identification of seasonal patterns, anomalous years, and long-term trends. The seasonal pattern is clearly observed (higher values in May to November months) and a general trend toward darker colors (lower precipitation) in more recent years, particularly in the rainy season.}
\label{fig:heatmap}
\end{figure}

The heatmap (Figure \ref{fig:heatmap}) provides a comprehensive view of patterns:

\begin{enumerate}
    \item \textbf{Seasonality:} Clear horizontal band showing higher precipitation in rainy season months.
    \item \textbf{Interannual variability:} Different years show different color intensities, indicating variability.
    \item \textbf{Trends:} Gradual transition toward darker colors (lower precipitation) in recent years.
    \item \textbf{Extreme events:} Cells with very intense colors (positive or negative) indicate exceptional months.
\end{enumerate}

\subsubsection{Annual Precipitation Distribution}

\begin{figure}[H]
\centering
\includegraphics[width=0.95\textwidth]{DonBosco_Climate/plots/boxplot_precipitacion_anual.png}
\caption{Boxplot (box and whisker diagram) of annual precipitation. Each box represents the distribution of monthly precipitation for a specific year. The central line of the box indicates the median, the box edges show the first and third quartiles (Q1 and Q3), and the whiskers extend to 1.5 times the interquartile range. Points outside the whiskers represent outliers. This graph allows comparison of precipitation distribution between years and identification of years with unusual characteristics. A general reduction in annual median and greater variability in recent years is observed.}
\label{fig:boxplot}
\end{figure}

The annual boxplot (Figure \ref{fig:boxplot}) shows:

\begin{itemize}
    \item \textbf{Median trends:} Gradual reduction in annual precipitation median.
    \item \textbf{Variability:} Increase in interannual variability in recent years.
    \item \textbf{Outliers:} Years with exceptionally high or low precipitation.
    \item \textbf{Ranges:} Comparison of interquartile ranges between years.
\end{itemize}

\subsubsection{Correlation with ENSO}

\begin{figure}[H]
\centering
\includegraphics[width=0.95\textwidth]{DonBosco_Climate/plots/matriz_correlacion.png}
\caption{Correlation matrix between climate variables. This graph shows Pearson correlation coefficients between different variables (precipitation, average temperature, maximum temperature, minimum temperature). Values range from -1 (perfect negative correlation) to +1 (perfect positive correlation), with colors indicating the strength and direction of correlation. A negative correlation between precipitation and temperature is observed, which is expected in tropical climates (drier periods tend to be warmer). Correlation with ENSO indices (if included) shows the influence of global climate phenomena on local precipitation.}
\label{fig:correlation}
\end{figure}

The correlation matrix (Figure \ref{fig:correlation}) reveals important relationships:

\begin{enumerate}
    \item \textbf{Precipitation vs. Temperature:} Negative correlation, consistent with tropical climate patterns.
    \item \textbf{Temperatures among themselves:} Strong positive correlations between average, maximum, and minimum temperatures.
    \item \textbf{ENSO:} Significant correlation with ENSO indices, indicating influence of global phenomena.
\end{enumerate}

\subsubsection{Geographic Location}

\begin{figure}[H]
\centering
\includegraphics[width=0.95\textwidth]{DonBosco_Climate/plots/ubicacion_panama.png}
\caption{Geographic location map of Don Bosco, Villas de Andalucía, Panama. The map shows the exact position of the study area (marked with a red point) in the geographic context of Panama and Central America. Coordinates are 8.9824°N, 79.5199°W. This location is on the Pacific slope of Panama, in the metropolitan area of Panama City. Geographic position is relevant because the Pacific slope generally receives less precipitation than the Caribbean slope due to orographic effects and wind patterns. Additionally, location in a dense urban zone is important for understanding urbanization effects on local climate.}
\label{fig:location}
\end{figure}

Figure \ref{fig:location} geographically contextualizes the study:

\begin{itemize}
    \item \textbf{Location in Panama:} Metropolitan area of Panama City.
    \item \textbf{Pacific slope:} Explanation of relatively lower precipitation patterns.
    \item \textbf{Urban context:} Densely urbanized area, relevant for heat island effects.
    \item \textbf{Elevation:} Low elevation (30-50 m), without significant orographic effects.
\end{itemize}

\subsection{Drought Analysis}

Drought detection analysis identified multiple drought periods during the study period. Results show:

\begin{itemize}
    \item \textbf{Frequency:} Total number of drought months and percentage of total period.
    \item \textbf{Duration:} Consecutive drought periods, some extending several months.
    \item \textbf{Intensity:} Magnitude of deviation from historical average during droughts.
    \item \textbf{Trends:} Increase in drought frequency in recent years.
\end{itemize}

\subsection{Correlation with Global Phenomena}

Correlation analysis with ENSO indices reveals:

\begin{itemize}
    \item \textbf{ONI correlation:} Significant Pearson correlation coefficient.
    \item \textbf{Significance:} p-value indicating statistical significance.
    \item \textbf{Interpretation:} During El Niño events (positive ONI), precipitation tends to decrease, while La Niña (negative ONI) is associated with increased precipitation.
\end{itemize}

% ============================================
% DISCUSSION
% ============================================
\section{Discussion}
\label{sec:discussion}

\subsection{Factors Explaining Precipitation Reduction}

Based on data analysis and literature review, multiple factors are identified that explain why Don Bosco, Villas de Andalucía, receives less precipitation compared to other regions and shows a reduction trend:

\subsubsection{1. Urban Heat Island Effect (UHI)}

Accelerated urbanization in Don Bosco has created conditions that favor the urban heat island effect:

\begin{enumerate}
    \item \textbf{Impermeable surfaces:} Increased pavement and construction reduces the surface's capacity to absorb and retain water, decreasing local evapotranspiration.
    
    \item \textbf{Radiation absorption:} Construction materials (concrete, asphalt) have greater solar radiation absorption capacity than vegetation, raising local temperatures.
    
    \item \textbf{Vegetation reduction:} Loss of trees and green areas eliminates an important source of evapotranspiration, reducing available humidity in the local atmosphere.
    
    \item \textbf{Anthropogenic heat emission:} Human activities (traffic, air conditioning, industry) generate additional heat that contributes to local warming.
\end{enumerate}

\textbf{Physical mechanism:} Urban warming can create a local high-pressure "dome" that inhibits cloud formation and precipitation. Additionally, warmer air can "absorb" more moisture before reaching the saturation point necessary for precipitation formation.

\subsubsection{2. Change in Surface Roughness}

Urbanization significantly alters surface roughness:

\begin{itemize}
    \item \textbf{Tall buildings:} Create obstacles that alter wind patterns.
    \item \textbf{Reduced convergence:} Wind alteration can reduce air mass convergence necessary for convective precipitation formation.
    \item \textbf{Change in local circulation:} Buildings can create "rain shadows" where precipitation is reduced.
\end{itemize}

\subsubsection{3. Reduced Evapotranspiration}

Vegetation loss has a direct impact:

\begin{itemize}
    \item \textbf{Less local atmospheric moisture:} Without plant evapotranspiration, there is less water vapor available to form clouds and precipitation.
    \item \textbf{Altered hydrological cycle:} The local water cycle is interrupted when vegetation is replaced by impermeable surfaces.
    \item \textbf{Reduced condensation nuclei:} Some plants emit particles that act as cloud condensation nuclei (CCN), facilitating precipitation formation.
\end{itemize}

\subsubsection{4. Influence of Global Climate Phenomena}

Correlation analysis shows significant ENSO influence:

\begin{itemize}
    \item \textbf{El Niño frequency:} If the frequency or intensity of El Niño events has increased, this would explain more frequent drought periods.
    \item \textbf{Interaction with climate change:} Climate change may be altering ENSO patterns, affecting precipitation in Panama.
    \item \textbf{Increased variability:} Greater variability in ENSO events can lead to more extreme periods of drought and rain.
\end{itemize}

\subsubsection{5. Global Climate Change}

Global factors also contribute:

\begin{enumerate}
    \item \textbf{Alteration of circulation patterns:} Climate change is altering global atmospheric circulation patterns, affecting how and where precipitation forms.
    
    \item \textbf{Expansion of the Intertropical Convergence Zone (ITCZ):} Changes in ITCZ position and strength can affect precipitation in Panama.
    
    \item \textbf{Global temperature increase:} Higher temperatures can increase the air's capacity to retain moisture, but can also alter cloud formation patterns.
    
    \item \textbf{Changes in wind patterns:} Alterations in trade winds can affect precipitation on the Pacific slope.
\end{enumerate}

\subsubsection{6. Geographic and Orographic Factors}

The specific location also plays a role:

\begin{itemize}
    \item \textbf{Pacific slope:} Don Bosco is on the Pacific slope, which typically receives less precipitation than the Caribbean slope due to orographic effects.
    
    \item \textbf{Rain shadow:} Although elevation is low, relative position to mountain systems can create rain shadow effects.
    
    \item \textbf{Distance to ocean:} Distance and relative direction to moisture sources (oceans) affects water vapor availability.
\end{itemize}

\subsection{Synergy of Factors}

It is important to note that these factors do not act in isolation, but interact synergistically:

\begin{enumerate}
    \item \textbf{Amplification:} The UHI effect can amplify the effects of global climate change, and vice versa.
    
    \item \textbf{Positive feedback:} Less precipitation $\rightarrow$ less vegetation $\rightarrow$ more warming $\rightarrow$ less precipitation.
    
    \item \textbf{Extreme events:} The combination of factors can make extreme events (droughts) more frequent and intense.
\end{enumerate}

\subsection{Comparison with Other Regions}

The reduction observed in Don Bosco is more pronounced than in other regions of Panama due to:

\begin{itemize}
    \item \textbf{Higher degree of urbanization:} Compared to rural or semi-rural areas.
    \item \textbf{Specific location:} Combination of geographic and urban factors.
    \item \textbf{Development timing:} Accelerated urbanization in recent decades coincides with the period of observed reduction.
\end{itemize}

\subsection{Study Limitations}

It is important to recognize the limitations:

\begin{enumerate}
    \item \textbf{Data:} Dependence on weather station data that may not fully capture spatial variability.
    
    \item \textbf{Study period:} 25 years is relatively short to establish definitive climate trends, although sufficient to identify patterns.
    
    \item \textbf{Unmeasured factors:} Some factors (aerosols, specific land use changes) are not fully quantified.
    
    \item \textbf{Correlation vs. causality:} Identified correlations do not prove direct causality, although physical mechanisms are consistent.
\end{enumerate}

\subsection{Implications}

Findings have several important implications:

\begin{itemize}
    \item \textbf{Water resources:} Precipitation reduction affects water availability, requiring better management.
    
    \item \textbf{Urban planning:} Need to incorporate climate considerations in urban development.
    
    \item \textbf{Adaptation:} Adaptation strategies necessary to face changes in precipitation patterns.
    
    \item \textbf{Future research:} Need for more detailed studies on specific mechanisms and future projections.
\end{itemize}

% ============================================
% CONCLUSIONS
% ============================================
\section{Conclusions}
\label{sec:conclusions}

\subsection{Main Conclusions}

Based on comprehensive analysis of climate data for the period 2000-2025, the following main conclusions can be drawn:

\begin{enumerate}
    \item \textbf{Confirmed reduction trend:} There is a statistically significant trend of reduction in monthly precipitation in Don Bosco, Villas de Andalucía, during the study period. This trend is consistent with global and regional observations of changes in precipitation patterns.
    
    \item \textbf{Multiple factors:} Precipitation reduction cannot be attributed to a single factor, but results from the interaction of multiple factors, including:
    \begin{itemize}
        \item Urban heat island effect due to accelerated urbanization
        \item Reduced evapotranspiration due to vegetation loss
        \item Influence of global climate phenomena (ENSO)
        \item Global climate change
        \item Geographic factors specific to the location
    \end{itemize}
    
    \item \textbf{Correlation with ENSO:} A significant correlation was identified between local precipitation and ENSO indices (ONI), confirming the influence of global climate phenomena on local precipitation patterns.
    
    \item \textbf{Increased droughts:} Drought detection analysis reveals an increase in the frequency of drought periods, particularly in the second half of the study period (after 2010).
    
    \item \textbf{Increased variability:} In addition to reduction in average precipitation, an increase in interannual variability is observed, suggesting greater instability in climate patterns.
    
    \item \textbf{More frequent anomalies:} Anomaly analysis shows an increase in the frequency of negative anomalies (droughts) compared to positive anomalies, particularly in recent years.
\end{enumerate}

\subsection{Scientific Explanation}

The reduction in precipitation in Don Bosco, Villas de Andalucía, can be explained through the following interconnected physical mechanisms:

\begin{enumerate}
    \item \textbf{Heat island mechanism:} Urbanization has created local warming conditions that can inhibit precipitation formation through:
    \begin{itemize}
        \item Creation of a local high-pressure dome
        \item Reduction in relative humidity due to greater capacity of hot air to retain vapor
        \item Alteration of local circulation patterns
    \end{itemize}
    
    \item \textbf{Reduced local humidity:} Vegetation loss reduces evapotranspiration, decreasing the amount of water vapor available locally for cloud and precipitation formation.
    
    \item \textbf{Convergence alteration:} Changes in surface roughness (buildings) alter wind patterns, potentially reducing air mass convergence necessary for convective precipitation.
    
    \item \textbf{Global influence:} Changes in global climate patterns, including variations in ENSO and anthropogenic climate change, contribute to locally observed trends.
    
    \item \textbf{Synergy:} These factors interact synergistically, creating an amplified effect that results in the observed precipitation reduction.
\end{enumerate}

\subsection{Recommendations}

Based on the findings of this study, the following recommendations are made:

\subsubsection{Recommendations for Water Resource Management}

\begin{enumerate}
    \item \textbf{Source diversification:} Develop alternative water sources to compensate for precipitation reduction.
    
    \item \textbf{Improved storage:} Improve rainwater collection and storage systems.
    
    \item \textbf{Conservation:} Implement water conservation and efficient use programs.
    
    \item \textbf{Monitoring:} Establish continuous monitoring systems for water resources and precipitation patterns.
\end{enumerate}

\subsubsection{Recommendations for Urban Planning}

\begin{enumerate}
    \item \textbf{Green infrastructure:} Incorporate green infrastructure (parks, green roofs, infiltration areas) to mitigate heat island effects and increase evapotranspiration.
    
    \item \textbf{Green spaces:} Increase and preserve green spaces and urban vegetation.
    
    \item \textbf{Permeable surfaces:} Use permeable materials in construction to allow infiltration and reduce runoff.
    
    \item \textbf{Climate planning:} Incorporate climate considerations in urban development plans.
\end{enumerate}

\subsubsection{Recommendations for Future Research}

\begin{enumerate}
    \item \textbf{Climate models:} Develop high-resolution regional climate models for future projections.
    
    \item \textbf{Detailed measurements:} Establish denser measurement networks to capture spatial variability.
    
    \item \textbf{Mechanism studies:} Investigate in detail specific physical mechanisms through numerical modeling.
    
    \item \textbf{Regional comparison:} Conduct comparative studies with other urban areas of Panama and the region.
    
    \item \textbf{Projections:} Develop future precipitation projections under different climate change and urban development scenarios.
\end{enumerate}

\subsubsection{Recommendations for Public Policy}

\begin{enumerate}
    \item \textbf{Urban regulation:} Develop regulations requiring incorporation of green infrastructure in new developments.
    
    \item \textbf{Vegetation conservation:} Establish policies to protect and increase urban vegetation.
    
    \item \textbf{Climate adaptation:} Develop climate change adaptation plans at local and regional levels.
    
    \item \textbf{Education:} Implement education programs on climate change and water resource management.
\end{enumerate}

\subsection{Study Contributions}

This study contributes to scientific knowledge in several aspects:

\begin{enumerate}
    \item \textbf{Data:} Provides detailed quantitative analysis of precipitation patterns in a specific area of Panama.
    
    \item \textbf{Methodology:} Demonstrates application of advanced statistical methods for climate analysis.
    
    \item \textbf{Understanding:} Increases understanding of how local factors (urbanization) and global factors (climate change, ENSO) interact to affect climate patterns.
    
    \item \textbf{Visualization:} Provides high-quality visualizations that facilitate understanding of complex patterns.
    
    \item \textbf{Base for future studies:} Establishes a database and methodology for future studies and comparisons.
\end{enumerate}

\subsection{Final Considerations}

The observed precipitation reduction in Don Bosco, Villas de Andalucía, is a complex phenomenon resulting from the interaction of multiple factors. Understanding these factors and their interactions is crucial for:

\begin{itemize}
    \item Developing effective adaptation strategies
    \item Planning sustainable urban development
    \item Managing water resources efficiently
    \item Mitigating negative impacts of climate change
\end{itemize}

This study provides quantitative and qualitative evidence that can inform public policy decisions, urban planning, and resource management. However, continuous research is required to refine understanding of specific mechanisms and develop reliable projections for the future.

% ============================================
% REFERENCES
% ============================================
\section{References}
\label{sec:references}

\begin{thebibliography}{99}

\bibitem{noaa_enso}
NOAA Climate Prediction Center. (2025). \textit{El Niño/Southern Oscillation (ENSO) Diagnostic Discussion}. National Oceanic and Atmospheric Administration. Available at: \url{https://www.cpc.ncep.noaa.gov/products/analysis_monitoring/ensostuff/ONI_change.shtml}

\bibitem{urban_precipitation}
Shepherd, J. M. (2005). A review of current investigations of urban-induced rainfall and recommendations for the future. \textit{Earth Interactions}, 9(12), 1-27.

\bibitem{ipcc_ar6}
IPCC. (2021). \textit{Climate Change 2021: The Physical Science Basis. Contribution of Working Group I to the Sixth Assessment Report of the Intergovernmental Panel on Climate Change}. Cambridge University Press.

\bibitem{enso_changes}
Cai, W., et al. (2015). ENSO and greenhouse warming. \textit{Nature Climate Change}, 5, 849-859.

\bibitem{panama_climate}
Aguilar, E., et al. (2005). Changes in precipitation and temperature extremes in Central America and northern South America, 1961-2003. \textit{Journal of Geophysical Research: Atmospheres}, 110(D23).

\bibitem{uhi_effects}
Oke, T. R. (1982). The energetic basis of the urban heat island. \textit{Quarterly Journal of the Royal Meteorological Society}, 108(455), 1-24.

\bibitem{precipitation_trends}
Trenberth, K. E., et al. (2007). Observations: Surface and Atmospheric Climate Change. In: \textit{Climate Change 2007: The Physical Science Basis}. Contribution of Working Group I to the Fourth Assessment Report of the IPCC.

\bibitem{statistical_methods}
Wilks, D. S. (2011). \textit{Statistical Methods in the Atmospheric Sciences} (3rd ed.). Academic Press.

\bibitem{stl_decomposition}
Cleveland, R. B., Cleveland, W. S., McRae, J. E., \& Terpenning, I. (1990). STL: A seasonal-trend decomposition procedure based on loess. \textit{Journal of Official Statistics}, 6(1), 3-73.

\bibitem{correlation_analysis}
Pearson, K. (1896). Mathematical contributions to the theory of evolution. III. Regression, heredity, and panmixia. \textit{Philosophical Transactions of the Royal Society of London}, 187, 253-318.

\bibitem{panama_urbanization}
Heckadon-Moreno, S. (2004). \textit{Panamá: Urban Growth and Sustainable Development}. Smithsonian Tropical Research Institute.

\bibitem{climate_data_analysis}
Kalnay, E., et al. (1996). The NCEP/NCAR 40-year reanalysis project. \textit{Bulletin of the American Meteorological Society}, 77(3), 437-471.

\bibitem{noaa_data}
NOAA National Centers for Environmental Information. (2025). \textit{Global Summary of the Day (GSOD)}. Available at: \url{https://www.ncei.noaa.gov/access/services/data/v1}

\bibitem{urban_climate}
Arnfield, A. J. (2003). Two decades of urban climate research: a review of turbulence, exchanges of energy and water, and the urban heat island. \textit{International Journal of Climatology}, 23(1), 1-26.

\bibitem{precipitation_mechanisms}
Houze, R. A. (2014). \textit{Clouds and Precipitation in the Tropics}. In: \textit{Clouds and Storms: The Behavior and Effect of Water in the Atmosphere}. Pennsylvania State University Press.

\bibitem{enso_mechanisms}
McPhaden, M. J., Zebiak, S. E., \& Glantz, M. H. (2006). ENSO as an integrating concept in earth science. \textit{Science}, 314(5806), 1740-1745.

\bibitem{climate_change_impacts}
Field, C. B., et al. (2014). \textit{Climate Change 2014: Impacts, Adaptation, and Vulnerability}. Contribution of Working Group II to the Fifth Assessment Report of the IPCC. Cambridge University Press.

\bibitem{statistical_trends}
Mann, H. B. (1945). Nonparametric tests against trend. \textit{Econometrica}, 13(3), 245-259.

\bibitem{anomaly_detection}
Hawkins, D. M. (1980). \textit{Identification of Outliers}. Chapman and Hall.

\bibitem{panama_water_resources}
Panama Canal Authority. (2020). \textit{Water Master Plan}. Panama.

\end{thebibliography}

% ============================================
% APPENDICES
% ============================================
\begin{appendices}

\section{Python Code for Data Analysis}
\label{app:code}

\subsection{Main Script (main.py)}

\begin{lstlisting}
"""
Main script for climate analysis of Don Bosco, Villas de Andalucía.
Studies climate behavior and precipitation variations (2000-2025).
"""

import sys
from datetime import datetime
from config import START_YEAR, END_YEAR, LOCATION, PROCESSED_DATA_DIR, PLOTS_DIR

# Import consolidated modules
from climate_data import (
    fetch_noaa_data, fetch_enso_indices, clean_climate_data,
    aggregate_to_monthly, save_data, calculate_statistics,
    calculate_precipitation_trends, detect_anomalies_zscore,
    correlate_with_enso, detect_drought_periods, calculate_annual_statistics
)
from visualization import (
    plot_time_series, plot_precipitation_anomalies, plot_enso_correlation,
    plot_heatmap_monthly_precipitation, plot_annual_boxplot,
    plot_correlation_matrix, plot_location_map
)

def main():
    """Main function."""
    print("=" * 80)
    print("CLIMATE ANALYSIS: DON BOSCO, VILLAS DE ANDALUCÍA (2000-2025)")
    print("=" * 80)
    print(f"\nDate: {datetime.now().strftime('%Y-%m-%d %H:%M:%S')}")
    print(f"Location: {LOCATION['name']} ({LOCATION['latitude']}, {LOCATION['longitude']})")
    print(f"Period: {START_YEAR} - {END_YEAR}\n")
    
    # Step 1: Data collection and processing
    print("-" * 80)
    print("STEP 1: DATA COLLECTION AND PROCESSING")
    print("-" * 80)
    
    print("\n1.1. Downloading climate data...")
    df_climate = fetch_noaa_data(START_YEAR, END_YEAR, LOCATION)
    print(f"    ✓ {len(df_climate)} daily records")
    
    print("\n1.2. Downloading ENSO indices...")
    df_enso = fetch_enso_indices(START_YEAR, END_YEAR)
    print(f"    ✓ {len(df_enso)} monthly records")
    
    print("\n1.3. Cleaning and processing data...")
    df_clean = clean_climate_data(df_climate)
    df_monthly = aggregate_to_monthly(df_clean)
    print(f"    ✓ {len(df_clean)} clean daily records")
    print(f"    ✓ {len(df_monthly)} monthly records")
    
    print("\n1.4. Saving processed data...")
    save_data(df_clean, "climate_data_daily")
    save_data(df_monthly, "climate_data_monthly")
    save_data(df_enso, "enso_indices")
    print("    ✓ Data saved")
    
    # Step 2: Statistical analysis
    print("\n" + "-" * 80)
    print("STEP 2: STATISTICAL ANALYSIS")
    print("-" * 80)
    
    print("\n2.1. Basic statistics...")
    stats_precip = calculate_statistics(df_monthly, 'precipitation_mm')
    print(f"    ✓ Average precipitation: {stats_precip['mean']:.2f} mm/month")
    print(f"    ✓ Standard deviation: {stats_precip['std']:.2f} mm")
    print(f"    ✓ Range: {stats_precip['min']:.2f} - {stats_precip['max']:.2f} mm")
    
    print("\n2.2. Trend analysis...")
    trends = calculate_precipitation_trends(df_monthly)
    print(f"    ✓ Total change: {trends['change_total_percent']:.2f}%")
    print(f"    ✓ Annual change: {trends['change_per_year_percent']:.2f}%/year")
    print(f"    ✓ R²: {trends['r_squared']:.4f}, p-value: {trends['p_value']:.4e}")
    
    print("\n2.3. Anomaly detection...")
    df_anomalies = detect_anomalies_zscore(df_monthly, 'precipitation_mm')
    n_anomalies = df_anomalies['anomaly_zscore'].sum()
    print(f"    ✓ Anomalies detected: {n_anomalies} ({100*n_anomalies/len(df_anomalies):.1f}%)")
    
    print("\n2.4. ENSO correlation...")
    enso_corr = correlate_with_enso(df_monthly, df_enso, 'precipitation_mm')
    print(f"    ✓ ONI correlation: {enso_corr['correlation']:.4f}")
    print(f"    ✓ p-value: {enso_corr['p_value']:.4f}")
    
    print("\n2.5. Drought detection...")
    df_drought = detect_drought_periods(df_monthly)
    n_drought = df_drought['drought'].sum()
    print(f"    ✓ Drought months: {n_drought} ({100*n_drought/len(df_drought):.1f}%)")
    
    # Step 3: Visualization
    print("\n" + "-" * 80)
    print("STEP 3: VISUALIZATION")
    print("-" * 80)
    
    plot_files = []
    
    print("\n3.1. Time series...")
    plot_time_series(df_monthly, 'date', 'precipitation_mm',
                    title='Monthly Precipitation - Don Bosco, Villas de Andalucía (2000-2025)',
                    ylabel='Precipitation (mm)', save_path='precipitacion_temporal.png')
    plot_files.append('precipitacion_temporal.png')
    
    print("3.2. Anomalies...")
    plot_precipitation_anomalies(df_monthly, save_path='anomalias_precipitacion.png')
    plot_files.append('anomalias_precipitacion.png')
    
    print("3.3. Monthly heatmap...")
    plot_heatmap_monthly_precipitation(df_monthly, save_path='heatmap_precipitacion_mensual.png')
    plot_files.append('heatmap_precipitacion_mensual.png')
    
    print("3.4. Annual boxplot...")
    plot_annual_boxplot(df_monthly, save_path='boxplot_precipitacion_anual.png')
    plot_files.append('boxplot_precipitacion_anual.png')
    
    print("3.5. ENSO correlation...")
    plot_enso_correlation(df_monthly, df_enso, save_path='correlacion_enso.png')
    plot_files.append('correlacion_enso.png')
    
    print("3.6. Location map...")
    fig = plot_location_map(save_path='ubicacion_panama.png')
    if fig is not None:
        plot_files.append('ubicacion_panama.png')
    
    if 'temperature_avg_c' in df_monthly.columns:
        print("3.7. Correlation matrix...")
        variables = ['precipitation_mm', 'temperature_avg_c']
        if 'temperature_max_c' in df_monthly.columns:
            variables.append('temperature_max_c')
        plot_correlation_matrix(df_monthly, variables,
                               title='Correlation between Climate Variables',
                               save_path='matriz_correlacion.png')
        plot_files.append('matriz_correlacion.png')
    
    print(f"\n    ✓ {len(plot_files)} plots generated")
    
    # Final summary
    print("\n" + "=" * 80)
    print("SUMMARY")
    print("=" * 80)
    print(f"✓ Data: {len(df_monthly)} monthly records processed")
    print(f"✓ Plots: {len(plot_files)} generated")
    print(f"✓ Trend: {trends['change_total_percent']:.2f}% ({'Significant' if trends['p_value'] < 0.05 else 'Not significant'})")
    print(f"✓ Files: {PROCESSED_DATA_DIR}/ and {PLOTS_DIR}/")
    print("=" * 80 + "\n")

if __name__ == "__main__":
    try:
        main()
    except KeyboardInterrupt:
        print("\n\nInterrupted by user.")
        sys.exit(1)
    except Exception as e:
        print(f"\n\nERROR: {e}")
        import traceback
        traceback.print_exc()
        sys.exit(1)
\end{lstlisting}

\subsection{Configuration (config.py)}

\begin{lstlisting}
"""
Centralized configuration for climate analysis of Don Bosco, Villas de Andalucía.
"""

# Coordinates of Don Bosco, Villas de Andalucía, Panama
LOCATION = {
    'latitude': 8.9824,  # Latitude of Don Bosco, Panama City
    'longitude': -79.5199,  # Longitude of Don Bosco, Panama City
    'name': 'Don Bosco, Villas de Andalucía, Panama'
}

# Year range for analysis
START_YEAR = 2000
END_YEAR = 2025

# Directory paths
DATA_DIR = "data"
RAW_DATA_DIR = "data/raw"
PROCESSED_DATA_DIR = "data/processed"
PLOTS_DIR = "plots"
NOTEBOOKS_DIR = "notebooks"

# API URLs and endpoints
NOAA_BASE_URL = "https://www.ncei.noaa.gov/access/services/data/v1"
NASA_EARTHDATA_BASE_URL = "https://cmr.earthdata.nasa.gov/search"
EM_DAT_BASE_URL = "https://public.emdat.be/api"

# Visualization parameters
PLOT_DPI = 300  # Resolution for plots
PLOT_FORMAT = 'png'  # Export format
FIG_SIZE = (12, 6)  # Standard figure size

# Analysis parameters
Z_SCORE_THRESHOLD = 2.5  # Threshold for anomaly detection using Z-Score
STL_SEASONAL = 12  # Seasonal period for STL decomposition (monthly)
\end{lstlisting}

\section{Processed Data}
\label{app:data}

Processed data are available in CSV and Parquet format in the \texttt{data/processed/} directory:

\begin{itemize}
    \item \texttt{climate\_data\_daily.csv/parquet}: Daily precipitation and temperature data
    \item \texttt{climate\_data\_monthly.csv/parquet}: Aggregated monthly data
    \item \texttt{enso\_indices.csv/parquet}: Monthly ENSO (ONI) indices
\end{itemize}

\section{Generated Figures}
\label{app:figures}

All figures are available in the \texttt{plots/} directory at 300 DPI resolution:

\begin{enumerate}
    \item \texttt{precipitacion\_temporal.png}: Monthly precipitation time series
    \item \texttt{anomalias\_precipitacion.png}: Detected precipitation anomalies
    \item \texttt{heatmap\_precipitacion\_mensual.png}: Precipitation heatmap by year and month
    \item \texttt{boxplot\_precipitacion\_anual.png}: Annual precipitation distribution
    \item \texttt{matriz\_correlacion.png}: Correlation matrix between climate variables
    \item \texttt{ubicacion\_panama.png}: Geographic location map
\end{enumerate}

\end{appendices}

\end{document}
